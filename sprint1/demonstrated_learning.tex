\documentclass[12pt, a4paper]{article}
\usepackage[T1]{fontenc}
\usepackage[utf8]{inputenc}
\usepackage{amsmath}
\usepackage{amssymb}
\usepackage{hyperref}
\usepackage{parskip}
\usepackage{float}
\usepackage{graphicx}
\usepackage{cleveref}
\usepackage{adjustbox}
\usepackage{color}
\usepackage{todonotes}
\hypersetup{
    colorlinks = true,
}

\title{Software Engineering TDT4140 \\ Group 26: Demonstrated learning of Core 1}

\author{Sindre Hansen \and Vegard Helgesen Hesselberg \and Eirik Rismyhr \and Stian Sørli}

\date{Spring 2017}

\begin{document}
\maketitle

\section{Teamwork}
\textbf{How have you worked together as a team?}

Since this first sprint was planned to be used on learning the frameworks and techniques we were supposed to use in the project, we did not spend much time properly working towards the product.
Not much time was spent together writing code and programming, instead we taught ourselves the frameworks individually. Despite this, we shared our knowledge and resources with each other.
There have been no arguments or disagreements within the team, which is a good thing.


\section{Techniques and methodologies}
\textbf{Which techniques, principles and methodologies (Scrum, XP, SEMAT, Quality assurance activities, etc.) have you used?}

As mentioned in the first paragraph, little time was spent programming, and thus little time to utilize these principles and techniques.
When planning, we have used principles from Scrum in the form of a prioritized backlog and iterative releases, together with Trello as a Kanban board.
Towards the end of the sprint we used pair programming, which we will also be using later in the project.


\section{Successes}
\textbf{What has worked for your team? Why?}

We've managed to get a solid understanding of Django and we understand how we will use it in the project. We have gained a good overview over the project and how we are going to solve it.
The group has been focused during work hours, and little time has been wasted doing nothing when we're supposed to work.


\section{Not so successes}
\textbf{What has not worked? Why not?}

We've struggled with being on time to meetings, and a fair amount of time was wasted on delays. This has partly been because of lack of discipline and the group's sense of professionalism. However, when we discussed this point, we decided to tighten the rules to discourage coming late.
We may have spent a bit too much time reading guides and documentation on Django. A better approach could have been jumping straight into the project and learning as we go.
Trello has not been used as much as it should, and was rarely updated with newer and finished backlog items.


\section{Improvements}
\textbf{Actionable items for improvement of performance.}

\begin{itemize}
    \item Being on time to meetings.
    \item More programming.
    \item More teamwork and cooperation.
    \item Better documentation.
\end{itemize}

\newpage
\section{Changes to backlog}
\textbf{Describe any changes to your backlog and why they were made.}

We changed the order on some of the more prioritized user stories to make more sense, so that we make the parts that don't depend on something else being done first before it can be implemented. We also changed the first user story to be more clear, and more accurately convey the intended functionality. 

\end{document}  