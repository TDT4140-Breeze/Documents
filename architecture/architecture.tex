\documentclass[12pt, a4paper]{article}
\usepackage[T1]{fontenc}
\usepackage[utf8]{inputenc}
\usepackage{amsmath}
\usepackage{amssymb}
\usepackage{hyperref}
\usepackage{parskip}
\usepackage{titlesec}
\usepackage{float}
\usepackage{graphicx}
\usepackage{cleveref}
\usepackage{adjustbox}
\usepackage{color}
\usepackage{todonotes}
\hypersetup{
    colorlinks = true,
}

\newcommand\fnurl[2]{%
  \href{#2}{#1}\footnote{\url{#2}}%
}

\titlespacing{\subsection}{0pt}{\parskip}{-4pt}

\setcounter{tocdepth}{1}

\title{TDT4140 -- Software Engineering \\ Group 26: Preliminary Architecture}

\author{Sindre Hansen \and Vegard Helgesen Hesselberg \and Eirik Rismyhr \and Stian Sørli}

\date{Spring 2017}

\begin{document}
\maketitle
\tableofcontents
\thispagestyle{empty}
\clearpage
\setcounter{page}{1}
\section{Scalable web app}
\subsection{Description}
Implementing the bot as a scalable web app
\subsection{Stakeholder concerns}
Concern from lecturers: Should work on both computers and smart phones. 
\subsection{Related user stories}
User story \# 9
\subsection{Solution}
Developing the bot as a scalable web app. Doing this, the bot will be easily accessed and utilized during lectures, as it's accessible through both laptops and mobile phones. 
\subsection{Considered alternative solutions}
\begin{itemize}
    \item Developing the bot as a native app running on either a computer or a mobile device.
    \item Developing the bot as a plugin to another existing application.
\end{itemize}
\subsection{Influencing forces}
\begin{minipage}{0.5\textwidth}
    \subsubsection*{Positive}
    \begin{itemize}
        \item Accessibility for the end users
        \item Ability to run on most devices without requiring extra development time
    \end{itemize}
\end{minipage}%
\begin{minipage}{0.5\textwidth}
    \subsubsection*{Negative}
    \begin{itemize}
        \item Variable internet connection can cause difficulties
        \item Application may not be well supported by outdated software
    \end{itemize}
\end{minipage}

\subsection{Evaluation}

\newpage
\section{MySQL as database}
\subsection{Description}
The chat logs will be saved in a MySQL database. 
\subsection{Stakeholder concerns}
What will be saved, how to get meaningful answers from the data.
\subsection{Related user stories}
User stories \# 3 and \# 4 
\subsection{Solution}

\subsection{Considered alternative solutions}
\begin{itemize}
    \item Data could otherwise be saved locally on the end users' device
    \item A remote non-database server to save the chatlogs
\end{itemize}
\subsection{Influencing forces}
\begin{minipage}{0.5\textwidth}
    \subsubsection*{Positive}
    \begin{itemize}
        \item Structured saving of valuable and useful data, letting the end users access and use it
        \item Easy to implement, as well as use for developers and end users
    \end{itemize}
\end{minipage}%
\begin{minipage}{0.5\textwidth}
    \subsubsection*{Negative}
    \begin{itemize}
        \item Possibility of database going down, causing issues for end users.
        \item Internet connection required to fetch data.
    \end{itemize}
\end{minipage}

\subsection{Evaluation}

\newpage
\section{Django for web framework}
\subsection{Description}
Django is a web framework based in python designed for quick creation and less code. It emphasises 
\subsection{Stakeholder concerns}

\subsection{Related user stories}

\subsection{Solution}

\subsection{Considered alternative solutions}
Writing the app in pure Python or html.

\subsection{Influencing forces}
\begin{minipage}{0.5\textwidth}
    \subsubsection*{Positive}
    \begin{itemize}
        \item Uses a lot of the developer's previous knowledge and skills
        \item Effective framework for developing the app we have in mind
    \end{itemize}
\end{minipage}%
\begin{minipage}{0.5\textwidth}
    \subsubsection*{Negative}
    \begin{itemize}
        \item Developers have to learn and understand the framework before implementing starts
    \end{itemize}
\end{minipage}

\subsection{Evaluation}

\newpage
\section{Tornado for web server}
\subsection{Description}
Using the Tornado framework for our servers.
\subsection{Stakeholder concerns}
Students chatting must be connected together, over a safe and fast connection.
\subsection{Related user stories}

\subsection{Solution}
For the system's communication, we will implement a Tornado web server, which allows for many open connections to be connected together, event-based.
\subsection{Considered alternative solutions}

\subsection{Influencing forces}
\begin{minipage}{0.5\textwidth}
    \subsubsection*{Positive}
    \begin{itemize}
        \item Event-based solution, making it suited for a chat program
    \end{itemize}
\end{minipage}%
\begin{minipage}{0.5\textwidth}
    \subsubsection*{Negative}
    \begin{itemize}
        \item Developers have to learn and understand the framework before implementing starts
    \end{itemize}
\end{minipage}

\subsection{Evaluation}

\newpage
\section{UI and general look}
\subsection{Description}

\subsection{Stakeholder concerns}

\subsection{Related user stories}

\subsection{Solution}

\subsection{Considered alternative solutions}

\subsection{Influencing forces}
\begin{minipage}{0.5\textwidth}
    \subsubsection*{Positive}
    \begin{itemize}
        \item Ting 1
    \end{itemize}
\end{minipage}%
\begin{minipage}{0.5\textwidth}
    \subsubsection*{Negative}
    \begin{itemize}
        \item Ting 1
    \end{itemize}
\end{minipage}

\subsection{Evaluation}

\end{document}