\appendix

\section{Product backlog}
\label{app:Product backlog}
\begin{tabular}{ c | p{0.7\textwidth} | c | c}
     ID & Story & Estimate & Priority \\ \hline
     %
     T1
     & As a lecturer I want to be able to create a "Super-room" where I can see all ongoing conversations connected to the current lecture.
     & X & 1 \\ \hline
     %
     T2
     & As a student I want to receive a code from the lecturer that I can enter into a website and be placed in a chatroom with other students. 
     & X & 2 \\ \hline
     %
     T3
     & As a lecturer I want to be able to read the chat logs so that I can monitor the discussions.
     & X & 3 \\ \hline
     %
     T4
     & As a student I want to be able to save the chat log to my personal device.
     & X & 4 \\ \hline
     %
     T5
     & As a lecturer I want to be able to create topics beforehand so that the groups have something to discuss.
     & X & 5 \\ \hline
     %
     T6
     & As a student I want to be able to request a new topic if the first one is completed or too difficult. 
     & X & 6 \\ \hline
     %
     T7
     & As a student I want to be able to send a message directly to the lecturer.
     & X & 7 \\ \hline
     %
     T8
     & As a lecturer/TA I want to be able to supervise chat rooms. 
     & X & 8 \\ \hline
     %
     T9
     & As a student I want to be able to use a laptop \emph{or} a smartphone for chatting. 
     & X & 9 \\ \hline
\end{tabular}

\newpage
\section{Activity plan}
\label{app:Activity plan}
\begin{adjustbox}{center}
\begin{tabular}{ p{0.1\paperwidth} | p{0.1\paperwidth} | p{0.4\textwidth} | p{0.1\paperwidth} | p{0.1\paperwidth} }
    Release due date & Story\# or other tasks & Description \newline \textcolor{red}{Remarks} & Estimated resource use & Actual resource use \\ \hline
    27.4.2017 & Project management and related & Initial project plan, wrap-ups \textcolor{red}{Project management, wrap-ups, meetings, coaching} & 11 h & 11 h \\ \cline{2-5}
     & Planning day &  & 35 h & 37 h \\ 
     
\end{tabular}
\end{adjustbox}

\newpage
\section{Risk assessment plan}
\label{app:Risk assessment}
\begin{adjustbox}{center}
\begin{tabular}{ p{0.1\paperwidth} | p{0.25\paperwidth} | p{0.3\paperwidth} }
    Risk & Means to prevent & Action and responsible \\ \hline
    Schedule slips and project cancellation
    & Daily meetings to check progress and short release cycles.
    & The programmers are responsible for dropping user stories if the schedule slips. The group leader leads the meetings and makes sure the group delivers the product on time. \\ \hline
    
    Business change, cost of changes
    & Take in new user stories in the planning meeting before each sprint. Automatic unit tests, no errors in final product.
    & The customer is responsible of informing the project manager when the requirements change. The manager has to decide which stories to include in the sprint. \\ \hline
    
    Defect rates
    & Automated unit tests, customer tests.
    & The \textbf{programmers} are responsible of creating unit tests which check that the code does what it is supposed to. The \textbf{customer} is responsible for testing the product during development and providing feedback to the developers. \\ \hline
    
    Business misunderstanding
    & Work closely with the customer.
    & The developers have to understand and analyze the needs of the customer, and divide these into user stories. The customer is required to give feedback to assure that the project is moving in the right direction. \\ \hline
    
    Technology
    & Sharing of knowledge and pair programming. 
    & \\ \hline
    
    Staff schedules
    & Weekly schedule planned in the start of the project
    & It is the project managers' responsibility to create a schedule that works for everyone and make sure that everyone shows up on time. If any problems occur he should inform the the customer. \\ \hline
    
\end{tabular}
\end{adjustbox}