\section{Problem}

\subsection{Opportunity}
For many years, university education has used the same format with frontal lectures, assignments and written exams. This way of teaching is outdated and poses several problems. 

We want to change this using the possibilities that computers and information technology provide.


\subsection{Stakeholder}
\begin{itemize}
    \item A lecturer at a university or a college.
    \item A student participating in a class or a lecture at a university or a college.
\end{itemize} 

\subsubsection{Persona}
A lecturer for a large group of student who wants to engage the students in discussions. He wants to be able to automatically divide students in groups where they will discuss predefined questions. To see what the students answered he wants to collect the answers from the groups. 

\subsection{Requirement}
\begin{itemize}
    \item The program we are to create has to be able to run as a standalone program or in tandem with other digital platforms.
    \item The solution has to manage online sources of input data.
    \item In order to be a functional bot, it must be able to process the online input data and generate output based on the input.
    \item For the ease of end users, the program must have a functional graphical user interface.
    \item When the program is finished, the bot will be distributed as open source software.
    \item While not a requirement, it is encouraged to use other known APIs and make use of open source software.
\end{itemize}

See \cref{app:Product backlog} for the initial product backlog.