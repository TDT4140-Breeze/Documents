\section{Solution}

\subsection{Deliverables}
\subsubsection{Software}
A functional roBOT, able to connect end users together in a chat room and interact with them based on the users' input. The roBOT will also gather all of the users' input as a chat log, which will be automatically sent to the lecturer and/or saved by the students.

\subsubsection{Infrastructure}
We are aiming for a web-based solution that will be able to run in both browser and on smartphones, although the smartphone solution is not highly prioritized. TODO

server, frontend som kan kommunisere med serveren, backend

\subsubsection{Data} 
TODO

\subsection{Work}
See \cref{app:Activity plan} for an activity plan with estimated working hours attached to each task. 

\subsection{Team}
\label{sec:Team}
\begin{adjustbox}{center}
\begin{tabular}{ l | l | l}
    Member & Role & Responsibility \\ \hline
    Eirik Rismyhr & & \\ 
    Sindre Hansen & & \\ 
    Stian Sørli & & \\ 
    Vegard Helgesen Hesselberg & Testwriter and tester & \\ 
    
\end{tabular}
\end{adjustbox}


\subsection{Way of working}
We will use git for version control of active development. For project related communication we will be using gitter, which is directly connected to github, although we also have a Facebook-chat for other communication. Our git repository can be found on github \fnurl{here}{https://github.com/sindrehan/TDT4140-Project}.

Quality assurance will be done as we go, trying to work in pairs as much as possible. This way we get a wider knowledge of the project as a whole. As the project goes on we will write tests as we discover what needs to be tested. 