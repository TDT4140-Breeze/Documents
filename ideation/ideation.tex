\documentclass[a4paper]{article}
\usepackage[T1]{fontenc}
\usepackage[utf8]{inputenc}
\usepackage{amsmath}
\usepackage{amssymb}
\usepackage{hyperref}
\usepackage{parskip}
\usepackage{float}
\usepackage{graphicx}
\usepackage{listings}
\usepackage{cleveref}
\usepackage{listings}

\title{Software Engineering TDT4140 - Concept Proposition}
\author{Sindre Hansen \and Vegard Helgesen Hesselberg \and Eirik Rismyhr \and Stian Sørli}
\date{Spring 2017}

\begin{document}
\maketitle
\rule{\linewidth}{0.5mm}

\section{Top five problems of university education}
We interviewed 3 professors, two from NTNU and one external. We studied our interview notes and found the 5 most prominent and important issues. Not all the problems are solvable by a bot, however.

\subsection{Understanding}
One of the hardest problems is to judge to which extent the subject matter is being taken in by the students. It is hard to know where they lie in terms of theoretical and practical knowledge, and where they have holes. This is a hard problem to solve, as it requires a lot of input from students and processing of that input by the teacher or teaching assistants. In subjects where the exercises have to be approved by student assistants (SA), the SAs can formulate an average of the skill level of the students and what points should be repeated in the lecture, and give that to the teacher and Teaching Assistants (TA).

\subsection{Varying levels of difficulty}
Large classes have students with widely different levels of experience and knowledge. Adjusting the difficulty level of the assignments is not easy, because they can’t be too easy for the experienced students or too difficult for the less experienced. A possible solution to this problem is to allow the students to choose tasks from several difficulty levels so that they get tasks that are suitable for them.

\subsection{Facilitating further discussion}
It is difficult to facilitate further discussion among students. Experience indicate that students need to be coerced into discussion. A possible solution to this problem could be to create a bot that creates student groups of 3 to 5 and places them in a chat room with a designated topic. The chat logs can then be sent to the teacher and/or TAs for evaluation.

\subsection{Practical and relevant goals}
Each course should have relevant exercises, so that students can acquire skills they could use when working in a business. One possibility is to assign the students a large project, which is divided into smaller exercises. This might reduce copying of solutions. Grading could be done on the whole project instead of individual exercises. This could replace a traditional written exam. This way of working is more similar to how it's done in the real world.

\subsection{Automation}
In the current system educators either spend a large amount of time creating new exercises or they reuse old ones, making it easy for students to copy old solutions. In technical classes this is a problem that can be solved fairly easily with automation. Exercises can be auto generated around a theme or based on a template exercise where parameters can be tweaked to adjust the difficulty. Automation could also be of great use in grading, particularly if the tasks already are auto generated.

\section{Interviews}
\subsection{Svein Sunde}
Professor, Department of Materials Science and Engineering. His areas of specialty include electrochemistry, water electrolysis and modeling of electrochemical systems. He is a course coordinator in TMT4115 -- General Chemistry and TMT4252 -- Electrochemistry.

\subsection{Frank Lindseth}
Associate Professor at IDI, lecturer in Computer vision (TDT4265) and visual computing fundamentals (TDT4195). Frank has worked both as an associate professor at IDI and as a Senior Researcher at the Medical Technology department of SINTEF since 2014, and has published many articles within the field of visual computing.

\subsection{Tom Heine Nätt} 
Works as a Senior Lecturer at University College in Østfold at their department of information science. He teaches mostly web-related subjects, with a heavy focus on computer security. He has ten years of experience in this position, and has written most of the curriculum he uses for his own classes, but also other publications on computer and web security.

\section{Early idea concept}
Our roBOT, named \textit{Breeze}, will solve, or at least lessen the effects of, the "Facilitating further discussion" problem. We believe that a chat where students discuss a topic is a good solution.

\appendix

\section{Product backlog}
\label{app:Product backlog}
\begin{tabular}{ c | p{0.7\textwidth} | c | c}
     ID & Story & Estimate & Priority \\ \hline
     %
     T1
     & As a lecturer I want to be able to create a "Super-room" where I can see all ongoing conversations connected to the current lecture.
     & 8 & 1 \\ \hline
     %
     T2
     & As a student I want to receive a code from the lecturer that I can enter into a website and be placed in a chatroom with other students. 
     & 10 & 2 \\ \hline
     %
     T3
     & As a lecturer I want to be able to read the chat logs so that I can monitor the discussions.
     & 3 & 3 \\ \hline
     %
     T4
     & As a student I want to be able to save the chat log to my personal device.
     & 3 & 4 \\ \hline
     %
     T5
     & As a lecturer I want to be able to create topics beforehand so that the groups have something to discuss.
     & 4 & 5 \\ \hline
     %
     T6
     & As a student I want to be able to request a new topic if the first one is completed or too difficult. 
     & 5 & 6 \\ \hline
     %
     T7
     & As a student I want to be able to send a message directly to the lecturer.
     & 7 & 7 \\ \hline
     %
     T8
     & As a lecturer/TA I want to be able to supervise chat rooms. 
     & 8 & 8 \\ \hline
     %
     T9
     & As a student I want to be able to use a laptop \emph{or} a smartphone for chatting. 
     & 4 & 9 \\ \hline
\end{tabular}

\newpage
\section{Activity plan}
\label{app:Activity plan}
\begin{adjustbox}{center}
\begin{tabular}{ p{0.1\paperwidth} | p{0.1\paperwidth} | p{0.4\textwidth} | p{0.1\paperwidth} | p{0.1\paperwidth} }
    %
    Release due date 
    & Tasks 
    & Description
    & Est. resource use (h)
    & Actual resource use (h)\\ \hline
    %
    23.01.17 
    & Ideation report
    & Understanding the purpose and goal of the project
    & -
    & 32 \\ \hline
    %
    03.02.17
    & Project Plan, Project Backlog
    & Pre-project documentation, gain an overview of the project
    & - 
    & 46 \\ \hline
    %
    10.02.17
    & Requirement and Architecture
    & 
    & 32 
    & - \\ \hline
    %
    24.02.17
    & Sprint 1
    & Get back-end up and running
    & 64 
    & - \\ \hline
    %
    10.03.17
    & Sprint 2
    & Must have the ability to create and join chatrooms by now, basic application functionality in place
    & 64 
    & - \\ \hline
    %
    24.03.17
    & Sprint 3
    & Expanding functionality of the roBOT, adding required features, phone functionality
    & 64
    & - \\ \hline
    %
    07.04.17
    & Sprint 4
    & Polish, bug removal and front-end fixing. Testing
    & 64
    & - \\ \hline
    %
    27.04.17
    & Final report
    & Documentation and preparation for the final presentation
    & 64
    & - \\ \hline
    %
    & Total
    & 
    & 334
    & -
\end{tabular}
\end{adjustbox}

\newpage
\section{Risk assessment plan}
\label{app:Risk assessment}
\begin{adjustbox}{center}
\begin{tabular}{ p{0.1\paperwidth} | p{0.25\paperwidth} | p{0.3\paperwidth} }
    Risk & Means to prevent & Action and responsible \\ \hline
    %
    Schedule slips and project cancellation
    & Daily meetings to check progress and short release cycles.
    & The \textbf{programmers} are responsible for dropping user stories if the schedule slips. The \textbf{group leader} leads the meetings and makes sure the group delivers the product on time. \\ \hline
    %
    Business change, cost of changes
    & Take in new user stories in the planning meeting before each sprint. Automatic unit tests, no errors in final product.
    & The \textbf{customer} is responsible of informing the project manager when the requirements change. The \textbf{project manager} has to decide which stories to include in the sprint. \\ \hline
    %
    Defect rates
    & Automated unit tests, customer tests.
    & The \textbf{programmers} are responsible of creating unit tests which check that the code does what it is supposed to. The \textbf{customer} is responsible for testing the product during development and providing feedback to the developers. \\ \hline
    %
    Business misunderstanding
    & Work closely with the customer.
    & The \textbf{developers} have to understand and analyze the needs of the customer, and divide these into user stories. The \textbf{customer} is required to give feedback to assure that the project is moving in the right direction. \\ \hline
    %
    Technology
    & Sharing of knowledge and pair programming. 
    & The \textbf{programmers} work in pairs to share knowledge and quality check the code. They inform the \textbf{project manager} if any problems occur, and he sets up code reviews and training sessions. \\ \hline
    %
    Staff schedules
    & Weekly schedule planned in the start of the project
    & It is the \textbf{project managers}' responsibility to create a schedule that works for everyone and make sure that everyone shows up on time. If any problems occur he should inform the the customer. \\ \hline
\end{tabular}
\end{adjustbox}
\end{document}