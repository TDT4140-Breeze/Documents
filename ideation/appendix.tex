\appendix
\section{Original response data}
\subsection{Svein Sunde}
\begin{enumerate}
    \item Knowing exactly how one should teach in a subject, down to which models to use, formulas and explanations.
    \item Presenting the goals of individual lectures.
    \item Achieving the correct relevance, quality and difficulty of exercises.
    \item Avoid students copying each other's answers. So-called "kok".
    \item Quality of educational area (both lectures and exercises).
\end{enumerate}

\subsection{Frank Lindseth}
\begin{enumerate}
    \item There's a difference between learning/understanding and repeating what's written in the textbook.
    \item Through automation of exercises, exams and correcting, lecturers will be able to manage their time better and spend it on dialogue and other more constructive activities.
    \item "Blackboard lecturing" is dying.
    \item Lecturer should have more of a facilitator and guide for the students instead of spending time lecturing.
    \item Students should practice what they learn, actively work on concrete problem solving.
\end{enumerate}

\subsection{Tom Heine Nätt} 
\begin{enumerate}    
    \item Large skill difference between students, exercises either end up too easy for some or too difficult for others.
    \item The division of subjects at their college often causes some pieces of the curriculum to be cut.
    \item Getting the best use of the 180 study points is a challenge, as informatics is such a wide field, it is difficult to decide whether to focus on basic knowledge or more advanced and specific subjects.
    \item It's not easy to motivate students to comprehend the width of informatics, programming students often ask why they need to understand network protocols, and data system workers wonder why they need to learn programming.
\end{enumerate}