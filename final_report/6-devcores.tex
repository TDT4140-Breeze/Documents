\section {Development Cores}
\subsection{Core 1 Retrospective}
\subsubsection{Teamwork}
Since this first sprint was planned to be used on learning the frameworks
and techniques we were supposed to use in the project, we did not spend
much time properly working towards the product. Not much time was spent
together writing code and programming, instead we taught ourselves the
frameworks individually. Despite this, we shared our knowledge and resources
with each other. There have been no arguments or disagreements within the
team, which is a good thing.

\subsubsection{Techniques and methodologies}
As mentioned in the first paragraph, little time was spent programming, and
thus little time to utilize these principles and techniques. When planning,
we have used principles from Scrum in the form of a prioritized backlog and
iterative releases, together with Trello as a Kanban board. Towards the end
of the sprint we used pair programming, which we will also be using later in
the project.

\subsubsection{What worked well}
We’ve managed to get a solid understanding of Django and we understand
how we will use it in the project. We have gained a good overview over
the project and how we are going to solve it. The group has been focused
during work hours, and little time has been wasted doing nothing when we’re
supposed to work.

\subsubsection{What did not work well}
We’ve struggled with being on time to meetings, and a fair amount of time
was wasted on delays. This has partly been because of lack of discipline and
the group’s sense of professionalism. However, when we discussed this point,
we decided to tighten the rules to discourage coming late. We may have spent
a bit too much time reading guides and documentation on Django. A better
approach could have been jumping straight into the project and learning as
we go. Trello has not been used as much as it should, and was rarely updated
with newer and finished backlog items.

\subsubsection{Improvements}
\begin{itemize}
    \item Being on time to meetings.
    \item More programming.
    \item More teamwork and cooperation.
    \item Better documentation.
\end{itemize}

\subsection{Core 2 Retrospective}
\subsubsection{Teamwork}
Like last sprint, we didn't really get that work done as a team. We were still learning, and hadn't started fully on the product.

\subsubsection{Techniques and methodologies}
Same as last sprint, we hadn't started fully on the product, so we didn't have much to use the techniques and methodologies on. This came back to bite us later.

\subsubsection{What worked well}
Team dynamic was good. We easily divided the work to prepare for next sprint. No quarreling. By now we had a good understanding of Django and the surrounding frameworks. 

\subsubsection{What did not work well}
Trello was not utilized very much. We had all the user stories there, but everyone worked on things that were \textit{related to} the user stories, and other non-functional requirements. Often times work would be done towards the user stories, but not exactly what was needed to consider it done by our definition. We believe splitting them up into different tasks more suited for programming piece by piece would let us utilize Trello more efficiently.

\subsubsection{Improvements}
\begin{itemize}
    \item Use Trello more efficiently
    \item Develop more on the product, more coding
    \item Meet on time
    \item Better documentation
\end{itemize}

\subsection{Core 3 Retrospective}
\subsubsection{Teamwork}
Teamwork was considerably better this sprint. Almost all work was done in pairs, with both groups talking about what they were doing. We really embraced pair programming, and almost all coding was done in discussion.

\subsubsection{Techniques and methodologies}
In this sprint we changed the skeleton of the code. This caused some extra work, but we think that we saved time and effort in the long run. We had to change some of how the database worked, but we got this up and running pretty fast. In the future we should start writing tests. 

\subsubsection{What worked well}
Teamwork was a big improvement from last sprint. The current pace seemed to be doing good for us. We have not had a need to work extra outside of the hours we set at the start.

\subsubsection{What did not work well}
Still not using Trello that much. No user stories were completely done. We had done a lot of work, but we were not able to set any of the user stories to Done. 

\subsubsection{Improvements}
\begin{itemize}
    \item Better plan for each sprint
    \item Use Trello more
    \item Be more precise in the user stories. Discuss what they encompass 
\end{itemize}

\subsection{Core 4 Retrospective}
\subsubsection{Teamwork}
We held teamwork at the same level as last sprint. We had clear communication, and saw improvements in code quality. We finally managed to use Trello in a productive way, and we got a visual representation of our progress, this again helped the team looking forward.

\subsubsection{Techniques and methodologies}
For this sprint we used many of the same methodologies as last sprint, in addition to the practice cards from the SEMAT Essence Kernel. As earlier we used continuous integration to have a working product as often as possible.

\subsubsection{What worked well}
We saw that we were going to complete the project with the product we had in mind. Even though we were going to have to do some work after Easter, we knew it was achievable.

\subsubsection{What did not work well}
Planning for this sprint didn't work as well as we would have wanted. We didn't get as much work done either. We saw that we were going to have to do some work on the product after Easter, but not too much as to be impossible.

\subsubsection{Improvements}
\begin{itemize}
    \item Work more
    \item Work more efficiently
    \item Better structure
    \item Better planning
\end{itemize}

\subsection{Summary}
By having these retrospective meetings we have improved our collaboration and communication a lot. This allowed us to work more efficiently on the product, which is the most important thing in this project. However, there were some core issues that proved difficult to fix. The lack of progress in Sprint 1 forced us to work more in the following Sprints, so we were always trying to work more efficiently. This caused some stress and we had to spend more than the estimated time to complete the product. We could likely have solved this by spending time before Sprint 1 to plan and learn Django. This could have given us more time to use on quality assurance. Despite this, we are happy with the way the product turned out.