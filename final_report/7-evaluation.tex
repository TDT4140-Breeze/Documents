\section {Evaluation and Conclusion}
\subsection{User feedback}
At the end of the project, we ran a small focus group to get feedback. Overall, that consensus was that the website was easy to navigate and easy to understand where you could go, but readability suffered a bit from poor choice of text- and background color. The users had no problem creating new lobbies. Some users reported an issue where it was difficult to spot the connection between the email and password fields when registering a new user.

\subsection{Delivery package}
The delivery package contains the latest code and documents from our repository. The code is taken from the master branch, so further development can be made by simply branching out from this. To run the software you have to follow the included instructions. This includes installing required software, starting the server and setting up a database. 

To make further development easier we have made sure to include comments in the functions and files to make the functionality easier to understand. The code is open source and freely available on our Github repository.

\subsection{Project evaluation}

\subsubsection{What have you learned from the project?}
\begin{itemize}
    \item How to use Django and surrounding software. Django is a widely used framework for web development, and having experience with this will be useful when developing other web sites and web based solutions.
    \item How to create a project plan. Having a well-structured plan is essential for a project of this size.   
    \item Collaboration and communication in a group. Throughout the project we have learned how important it is to communicate well within the group. Trello has been a useful collaboration tool which gave us an overview of what people were working on. This is something we can use later in similar projects.
    \item How to set up back-end for a software engineering project. This allowed us to use relevant knowledge we got from another course. We had to work both with scale and efficiency to get a good and working back-end system.
    \item How to make our own method using SEMAT essence kernel.
    \item The business side of Agile projects and documentation in Agile software development. Documentation is important for the project stakeholders and can support communication with an external group of developers. It doesn't have to be very comprehensive, it just needs to be good enough. Valuable development time can be wasted if too much time is spent on writing documentation.
    \item We learned how to design and build a software architecture. 
    \item Developing a back-end, a database, an interface and connecting these components to a working system. This proved to be one of most difficult tasks of the development process, but we learned a lot and got to use skills acquired from other relevant courses. We also learned the importance of having a clear architecture plan. This can prevent delays caused by lack of knowledge or unexpected issues.
\end{itemize}

\subsubsection{What is your suggestion for the next year course?}
\begin{itemize}
    \item Use what you have learned in your other courses (Human-Computer Interaction, Databases)
    \item Start working on the product early. Don't spend too much time learning frameworks. It's better to learn while working on the product.
    \item Plan well and follow the plan during the sprints.
    \item Do not forget to use all the resources and tools available to you, utilize them efficiently and properly.
\end{itemize}

\subsubsection{How we used SEMAT Essence Kernel in our project}
\begin{itemize}
    \item Positive aspects: It has allowed us to construct our own method using various practices. Having the ability to choose from existing methods is useful when creating your own method. We also liked having a kernel as a common ground.
    The SEMAT Kernel seems like something that gets better with consequent usage, if you already know what is applicable for a certain project, you can re-use your knowledge and apply the same practices to a similar project.
    \item Negative aspects: When creating a method it can be hard to choose from all the available practices, and it can be tempting to pick too many, instead of focusing on the ones that are the most useful for your group.
\end{itemize}