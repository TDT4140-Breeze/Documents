\section {Planning and Project Management}
\subsection{Project}
\subsubsection{Idea}
We started ideating after an interview with several lecturers, finding the most prominent challenges with the current lecturing system. The challenge we settled on was "Facilitating further discussion". We then started discussing platforms that could either solve or lessen the challenge. We finally chose to make a chat platform where the lecturer gives the students a code that when started, divides the students in rooms with a topic to discuss. 

\subsubsection{Team}
Our team consists of three second year computer science students and one fourth year Cybernetics student. 

\subsubsection{Stakeholders}
\begin{itemize}
    \item Our team
    \item Our project supervisor
    \item \textbf{Persona: Jonas Forel Eser}:
    
    Irritated at the technology he regularly uses which lets him down, and he has to use valuable lecturing time fixing said things. He wants a service to fill the downtime in lectures. 
    To make discussion between students healthier and more comfortable, he wishes for an application that can encourage students to discuss the topic currently being lectured. Instead of students having unproductive chatter, they could use such an application to discuss.
\end{itemize}

\subsubsection{Project roles}
\begin{adjustbox}{center}
\begin{tabular}{ l | l | l}
    Member & Role & Responsibility \\ \hline
    Eirik Rismyhr & Developer & Front-end\\ 
    Sindre Hansen & Developer & Front-end \\ 
    Stian Sørli & Team leader & Back-end\\ 
    Vegard Helgesen Hesselberg & Developer & Back-end\\ 
\end{tabular}
\end{adjustbox}

\subsection{Activity plan}
See \cref{sec:act.plan w resources} for activity plan with actual resource use.

\begin{adjustbox}{center}
\begin{tabular}{ p{0.1\paperwidth} | p{0.1\paperwidth} | p{0.4\textwidth} | p{0.1\paperwidth} }
    %
    Release due date 
    & Tasks 
    & Description
    & Est. \newline resource \newline use (h)
   \\ \hline
    %
    23.01.17 
    & Ideation report
    & Understanding the purpose and goal of the project
    & 32
    \\ \hline
    %
    03.02.17
    & Project Plan, Project Backlog
    & Pre-project documentation, gain an overview of the project
    & 46
    \\ \hline
    %
    10.02.17
    & Requirement and Architecture
    & Set up architecture plan and learn frameworks
    & 32 
    \\ \hline
    %
    24.02.17
    & Sprint 1
    & Get back-end up and running
    & 64 
    \\ \hline
    %
    10.03.17
    & Sprint 2
    & Must have the ability to create and join chat rooms by now, basic application functionality in place
    & 64 
    \\ \hline
    %
    24.03.17
    & Sprint 3
    & Expanding functionality of the roBOT, adding required features, phone functionality
    & 64
    \\ \hline
    %
    07.04.17
    & Sprint 4
    & Polish, bug removal and front-end fixing. Testing
    & 64
    \\ \hline
    %
    27.04.17
    & Final report
    & Documentation and preparation for the final presentation
    & 64
    \\ \hline
    %
    & Total
    & 
    & 412
    
\end{tabular}
\end{adjustbox}

\subsection{Risk assessment}
\begin{adjustbox}{center}
\begin{tabular}{ p{0.1\paperwidth} | p{0.25\paperwidth} | p{0.3\paperwidth} }
    Risk & Means to prevent & Action and responsible \\ \hline
    %
    Schedule slips and project cancellation
    & Daily meetings to check progress and short release cycles.
    & The \textbf{programmers} are responsible for dropping user stories if the schedule slips. The \textbf{group leader} leads the meetings and makes sure the group delivers the product on time. \\ \hline
    %
    Business change, cost of changes
    & Take in new user stories in the planning meeting before each sprint. Automatic unit tests, no errors in final product.
    & The \textbf{customer} is responsible of informing the project manager when the requirements change. The \textbf{project manager} has to decide which stories to include in the sprint. \\ \hline
    %
    Defect rates
    & Automated unit tests, customer tests.
    & The \textbf{programmers} are responsible of creating unit tests which check that the code does what it is supposed to. The \textbf{customer} is responsible for testing the product during development and providing feedback to the developers. \\ \hline
    %
    Business misunderstanding
    & Work closely with the customer.
    & The \textbf{developers} have to understand and analyze the needs of the customer, and divide these into user stories. The \textbf{customer} is required to give feedback to assure that the project is moving in the right direction. \\ \hline
    %
    Technology
    & Sharing of knowledge and pair programming. 
    & The \textbf{programmers} work in pairs to share knowledge and quality check the code. They inform the \textbf{project manager} if any problems occur, and he sets up code reviews and training sessions. \\ \hline
    %
    Staff schedules
    & Weekly schedule planned in the start of the project
    & It is the \textbf{project managers}' responsibility to create a schedule that works for everyone and make sure that everyone shows up on time. If any problems occur he should inform the customer. \\ \hline
    %
    
    
    
\end{tabular}
\end{adjustbox}

\begin{adjustbox}{center}
\begin{tabular}{ p{0.1\paperwidth} | p{0.25\paperwidth} | p{0.3\paperwidth} }
    Risk & Means to prevent & Action and responsible \\ \hline
    %
    Workload
    & Plan ahead and if necessary reduce the workload in the project.
    & The \textbf{developers} should not take on more work than they think they can get done. If necessary, the work can be dropped or reallocated to other people. \\ \hline
    %
    Integration issues
    & Run integration tests and collaborate
    & The people responsible for testing should make comprehensive integration tests and unit tests, so that the system works when connecting the components. \\ \hline
    %
    Bad team communication
    & Communicate when necessary and use task boards
    & The team members should inform each other of what they are doing and of any potential issues. Task boards can help the team keep track of what is being worked on. \\ \hline
\end{tabular}
\end{adjustbox}